\documentclass[a4paper,11pt,openright,openbib]{article}
\usepackage[portuges]{babel}
\usepackage[T1]{fontenc}
\usepackage{ae}
\usepackage[utf8]{inputenc}
\usepackage[pdftex]{graphicx}
\usepackage{url}
\usepackage{listings}
\usepackage{verbatim}
\usepackage{enumerate}
\usepackage[a4paper, pdftex, bookmarks, colorlinks, linkcolor=black, urlcolor=blue]{hyperref} 
\usepackage[a4paper,left=2.5cm,right=2.5cm,top=3.5cm,bottom=3.5cm]{geometry}
\usepackage{colortbl}
\usepackage[margin=10pt,font=small,labelfont=bf]{caption}
\usepackage{mdwlist}


\setlength{\parindent}{0cm}
\setlength{\parskip}{2pt}


%Como não sei de que formas costumas colocar os autores (se com o \author ou outro), não coloquei,
%mas depois podes colocar tu:
%--> António Silva, pg22820
%--> Rui Brito, pg.....


\title{
	\large{\includegraphics[width=0.3\textwidth]{../../report-template/UM}} \\
	\large{Universidade do Minho} \\
	\date{\today}
}


\begin{document}

\maketitle


\pagestyle{headings}
\pagenumbering{arabic}
\newpage
\tableofcontents
\newpage
%Ele disse que o relatório não precisa de ser uma coisa muito complexa. Essencialmente é para explicar o que foi feito,
%e no caso de haver respostas a dar, serem dadas aqui.
\section{Introdução}
Este primeiro trabalho de \emph{Engenharia Gramatical} de avaliação, da Unidade Curricular de Especialização 
\emph{Engenharia de Linguagens}, consiste na realização da ficha 3 e 5 disponiblizadas no Blackboard.
\section{Ficha 3}
\subsection{Composição do Corpo}
Para escrever uma gramática tradutora, foi necessário completar a informação sobre o corpo. Assim incialmente o corpo da
factura era um conjunto de linhas, em que cada linhas era \emph{'(' codartigo ',' designacao ',' pvu ',' quantidade ')'}.
Depois para suportar o pedido da alínea c, cada linhas passou a ser somente \emph{'(' codartigo ',' quantidade ')'}.
\subsection{Código Java}
Para conseguirmos saber os totais dos produtos, com várias facturas (sendo que cada factura possuia um id alfanumérico),
foi criado um \emph{hashmap} para associar a cada id de factura uma lista de valores, que era o total de cada linha
da factura. No final, é possível apresentar o o total de cada linha em cada factura e ainda o tal de cada factura (que 
mais não é que a soma dos totais das linhas).
Para se obter o Preço Unitário, que a pedido da alínea c) deveria já ter sido indicado no ínicio, foi também criada um
\emph{hashmap} com a correspondência entre o código do produto e os seus atributos (guardados numa classe). Assim, por
cada linha só era necessário obter o PVU através do código do artigo, e multiplicá-lo pela quantidade.
\subsection{Exemplo de Input}
\begin{verbatim}
a1 "xpto" 3.6 50
a2 "outro" 1 60.5
a3 "mais um" 4.99 4
---

f1
	"Nome 1" "NIF 1" "Morada 1" "NIB 1"
	"Nome 2" "NIF 2" "Morada 2"
	(a1,5) (a3,2)
;

f2
	"Nome 3" "NIF 3" "Morada 3" "NIB 3"
	"Nome 4" "NIF 4" "Morada 4"
	(a2,9.5) (a1,5) (a3,2)
;

f3
	"Nome 5" "NIF 5" "Morada 5" "NIB 5"
	"Nome 6" "NIF 6" "Morada 5"
	(a2,2.25)
.
\end{verbatim}
\section{Conclusão}
... %O que será para escrever
\end{document}
