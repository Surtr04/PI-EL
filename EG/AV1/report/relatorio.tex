\documentclass[a4paper,11pt,openright,openbib]{article}
\usepackage[portuges]{babel}
\usepackage[T1]{fontenc}
\usepackage{ae}
\usepackage[utf8]{inputenc}
\usepackage[pdftex]{graphicx}
\usepackage{url}
\usepackage{listings}
\usepackage{verbatim}
\usepackage{enumerate}
\usepackage[a4paper, pdftex, bookmarks, colorlinks, linkcolor=black, urlcolor=blue]{hyperref} 
\usepackage[a4paper,left=2.5cm,right=2.5cm,top=3.5cm,bottom=3.5cm]{geometry}
\usepackage{colortbl}
\usepackage[margin=10pt,font=small,labelfont=bf]{caption}
\usepackage{mdwlist}


\setlength{\parindent}{0cm}
\setlength{\parskip}{2pt}

\newcommand{\addimg}[3]{\begin{figure}[h!]
	\begin{center}
		\includegraphics[width=\textwidth,keepaspectratio]{#1}
	\end{center}	
	\caption{\label{#2}#3}
\end{figure}
}


\title{
	\large{\includegraphics[width=0.3\textwidth]{../../../report-template/UM.jpg}} \\
	\large{Universidade do Minho}  \\
	\large{Mestrado em Engenharia Informática}  \\
	\large{Engenharia de Linguagens}  \\
	\large{Engenharia Gramatical - Grupo 1}  \\	
	\large{\textbf{Resolução da Avaliação 1}} \\
	\large{Ano Lectivo de 2012/2013} \\
	\date{\today}
}

\author{	
	\begin{tabular}[t]{c c}      
        pg22820 - \textbf{António Silva} \\        
		pg22781 - \textbf{Rui Brito} \\   				
	\\ 
	\end{tabular}
}

\begin{document}

\maketitle


\pagestyle{headings}
\pagenumbering{arabic}
\newpage
\tableofcontents
\newpage

\section{Introdução}
O objectivo desta avaliação era desenvolver uma Gramática de Atributos, com recurso da ferramenta VisualLisa, para o exercício
já estudado da Lista não vazia.
\section{Resolução}
Essencialmente a resolução consistiu num \emph{ArrayList} com os vários conjuntos de somas que iam sendo passadas para cada
elemento e caso fosse inteiro e tivesse 3 ou mais caracteres antes.
\addimg{../imgs/NEList.png}{producoes}{Produções}
\\Na figura \ref{producoes} podemos ver todas as produções da Gramática de Atributos.

\subsection{Lista-Conteudo}
\addimg{../imgs/lista-conteudo.png}{lista_conteudo}{Produção 0: Lista - Conteudo}
Na figura \ref{lista_conteudo} podemos ver a constituição da primeira produção, enquanto na figura \ref{fn1_lista_conteudo}
podemos ver a inicialização dos atributos (que serão passados para o conteudo) assim como a obtenção do resultado.
\addimg{../imgs/fn1_lista-conteudo.png}{fn1_lista_conteudo}{Inicialização dos atributos herdados e obtenção do valor do atributo sintetizado.}

\subsection{Conteudo-Conteudo-Virg-Elemento}
\addimg{../imgs/conteudo-conteudo-virg-elemento.png}{conteudo_conteudo_virg_elemento}{Produção 1: Conteudo - Conteudo ',' Elemento}
Na figura \ref{conteudo_conteudo_virg_elemento} podemos ver a constituição da segunda produção, enquanto na figura 
\ref{fn1_conteudo_conteudo_virg_elemento} podemos ver o fluxo dos valores dos atributos (não existe neste ponto nenhum tipo
de alteração dos valores, são simplesmente copiados).
\addimg{../imgs/fn1_conteudo-conteudo-virg-elemento.png}{fn1_conteudo_conteudo_virg_elemento}{Transferência de valores entre os atributos.}

\subsection{Conteudo-Elemento}
\addimg{../imgs/conteudo-elemento.png}{conteudo_elemento}{Produção 2: Conteudo - Elemento}
Na figura \ref{conteudo_elemento} podemos ver a constituição da terceira produção (tal como a anterior, derivação de conteudo),
enquanto na figura \ref{fn1_conteudo_elemento} podemos ver o fluxo dos valores dos atributos (tal como na produção anterior
são simplesmente copiados).
\addimg{../imgs/fn1_conteudo-elemento.png}{fn1_conteudo_elemento}{Transferência de valores entre os atributos.}

\subsection{Elemento-String}
\addimg{../imgs/elemento-STRING.png}{elemento_string}{Produção 3: Elemento - STRING}
Na figura \ref{elemento_string} podemos ver a constituição da quarta produção, que necessita de mais computação dos
seus atributos como podemos ver respectivamente nas figuras \ref{fn1_elemento_STRING} e \ref{fn2_elemento_STRING}. Como
nesta produção o elemento é uma \emph{String}, então a contagem de caracteres é incrementada, e o caso de se ter chegado
ao terceiro caracter, é colocado o valor zero na nova \emph{slot} no \emph{ArrayList} das somas.
Assim a função \emph{maisUm} possuirá o seguinte código:
\begin{verbatim}return $1 + 1;\end{verbatim} %$
Enquanto a função \emph{add2Soma} será:
\begin{verbatim}if ($1 == 0) $2.add(-1); else {if ($1 == 2) $2.get($2.size()-1) = 0;} return $2;\end{verbatim} %$
Deste modo sempre que detectar o primeiro caracter adicionará um slot negativo para impedir futuras somas até haver
mais de três caracteres seguidos.
\addimg{../imgs/fn1_elemento-STRING.png}{fn1_elemento_STRING}{Adição na contagem de caracteres.}
\addimg{../imgs/fn2_elemento-STRING.png}{fn2_elemento_STRING}{Devolução do \emph{ArrayList} com as somas.}

\subsection{Elemento-Integer}
\addimg{../imgs/elemento-INTEGER.png}{elemento_integer}{Produção 4: Elemento - INTEGER}
Na figura \ref{elemento_integer} podemos ver a constituição da quinta produção, que  tal como a anterior necessita de 
mais computação dos seus atributos como podemos ver respectivamente nas figuras \ref{fn1_elemento_INTEGER} e 
\ref{fn2_elemento_INTEGER}. 
Assim a função \emph{eqchars} possuirá o seguinte código:
\begin{verbatim}return 0;\end{verbatim}
Enquanto a função \emph{addSoma} será:
\begin{verbatim}if ($1.get($1.size()-1) >= 0) $1.get($1.size()-1) += $2; return $1;\end{verbatim} %$
Deste modo o valor inteiro só será somado na última \emph{slot} se o número de caracteres imediatamente anterior for
maior ou igual a 3 (ou seja a \emph{slot} respectiva já tenha sida inicializada).
\addimg{../imgs/fn1_elemento-INTEGER.png}{fn1_elemento_INTEGER}{Reinicialização na contagem de caracteres.}
\addimg{../imgs/fn2_elemento-INTEGER.png}{fn2_elemento_INTEGER}{Devolução do \emph{ArrayList} com as somas.}

\section{Conclusão}
Infelizmente este trabalho foi um pouco mais pesaroso de realizar pelo facto de o VisualLisa fechar muitas vezes inexplicavelmente.

\end{document}
