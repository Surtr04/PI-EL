\documentclass[a4paper,11pt,openright,openbib]{article}
\usepackage[portuges]{babel}
\usepackage[T1]{fontenc}
\usepackage{ae}
\usepackage[utf8]{inputenc}
\usepackage[pdftex]{graphicx}
\usepackage{url}
\usepackage{listings}
\usepackage{verbatim}
\usepackage{enumerate}
\usepackage[a4paper, pdftex, bookmarks, colorlinks, linkcolor=black, urlcolor=blue]{hyperref} 
\usepackage[a4paper,left=2.5cm,right=2.5cm,top=3.5cm,bottom=3.5cm]{geometry}
\usepackage{colortbl}
\usepackage[margin=10pt,font=small,labelfont=bf]{caption}
\usepackage{mdwlist}


\setlength{\parindent}{0cm}
\setlength{\parskip}{2pt}




\title{
	\large{\includegraphics[width=0.3\textwidth]{../../../report-template/UM.jpg}} \\
	\large{Universidade do Minho}  \\
	\large{Mestrado em Engenharia Informática}  \\
	\large{Engenharia de Linguagens}  \\
	\large{Engenharia Gramatical - Grupo 1}  \\	
	\large{\textbf{Resolução da Avaliação 1}} \\
	\large{Ano Lectivo de 2012/2013} \\
	\date{\today}
}

\author{	
	\begin{tabular}[t]{c c}      
        pg22820 - \textbf{António Silva} \\        
		pg22781 - \textbf{Rui Brito} \\   				
	\\ 
	\end{tabular}
}

\begin{document}

\maketitle


\pagestyle{headings}
\pagenumbering{arabic}
\newpage
\tableofcontents
\newpage

\section{Introdução}
\section{Resolução}
\subsecion{Lista-Conteudo}
\subsecion{Conteudo-Conteudo_Elemento}
\subsecion{Conteudo-Elemento}
\subsecion{Elemento-String}
\subsecion{Elemento-Integer}
%Colocar as imagens nos vários locais com uma pequena explicação, não há muito a dizer, só mesmo nas duas 
%últimas produções é que nécessário explicar mais um bocado.
\section{Conclusão}

\end{document}
