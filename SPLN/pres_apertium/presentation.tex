\documentclass{beamer}

\usepackage{ucs}
\usepackage[utf8x]{inputenc}
\usepackage[T1]{fontenc}
\usepackage[english]{babel}
\usepackage[retainorgcmds]{IEEEtrantools}%	IEEEeqnarray
\usepackage{mathabx}%	convolution symbol
\usepackage{multi row}
\usepackage{epstopdf}
\usepackage{listings}


%	presentation info
\title{Apertium - An Open-Source Platform for Automatic Translation}

\author{António Silva, Rui Brito}

\institute[pg22820, pg22781]{
	Universidade do Minho
}

\date{Braga, Março 2013}


%	beamer options
\usetheme{Frankfurt}


\begin{document}%	begin presentation

\maketitle%	title slide

\begin{frame}
	\frametitle{Index}
	\tableofcontents
\end{frame}

\section{What it it}
\begin{frame}
	\begin{block}{What it is}
		\begin{itemize}
			\item Machine Translation Engine from the OpenTrad project			
			\item Designed to translate closely related languages
			\begin{itemize}
				\item Recently expanded to support more divergent languages (e.g. es-en)
			\end{itemize}
			\item Dictionaries come in language pairs
			\begin{itemize}
				\item Well defined XML format
			\end{itemize}
			\item Works with lttoolbox
				\begin{itemize}
					\item Toolbox for lexical processing, morphological analysis and word generation
			\end{itemize}
		\end{itemize}
	\end{block}
\end{frame}

\section{Installation}
\subsection{}
\begin{frame}[fragile]

	\begin{block}{Install dependencies}
		autoconf automake expat flex
		gettext gperf help2man libiconv libtool 
		libxml2 libxslt m4 ncurses p5-locale-gettext 
		pcre perl5 pkgconfig zlib gawk subversion
	\end{block}
	\begin{block}{Get sources}
		\lstset{
    		language=bash,
    		basicstyle=\ttfamily\small,
    		breaklines=true
		}
		\begin{lstlisting}
			$ svn co http://apertium.svn.sourceforge.net/svnroot/apertium/trunk/lttoolbox
			$ svn co http://apertium.svn.sourceforge.net/svnroot/apertium/trunk/apertium
		\end{lstlisting}
	\end{block}
	\begin{block}{Get Dictionary Pairs}
		\lstset{
    		language=bash,
    		basicstyle=\ttfamily\small,
    		breaklines=true
		}
		\begin{lstlisting}
			$ svn co http://apertium.svn.sourceforge.net/svnroot/apertium/trunk/apertium-es-pt
		\end{lstlisting}
		\tiny
		Full list of dictionary pairs available at: \url{http://wiki.apertium.org/wiki/List_of_dictionaries}
	\end{block}

\end{frame}

\subsection{}
\begin{frame}[fragile]	
			\begin{block}{Setting the environment}
				\lstset{
		    		language=bash,
		    		basicstyle=\ttfamily\small,
		    		breaklines=true
				}
				\begin{lstlisting}
					$ PKG_CONFIG_PATH=/usr/local/lib/pkgconfig
					$ export PKG_CONFIG_PATH			
				\end{lstlisting}
			\end{block}	
			For each downloaded package do:
			\begin{block}{Compilation}
				\lstset{
		    		language=bash,
		    		basicstyle=\ttfamily\small,
		    		breaklines=true
				}
				\begin{lstlisting}
					$ ./autogen.sh
					$ make
					$ sudo make install			
				\end{lstlisting}
			\end{block}			
\end{frame}

\section{Dictionary sample}
\begin{frame}[fragile]
	\scriptsize
	\begin{columns}[t]	
	\column{.5\textwidth}
	{\bf Sample XML:}
	\lstset{
		    		language=XML,
		    		basicstyle=\ttfamily\scriptsize,
		    		breaklines=true
				}
	\begin{lstlisting}
		<pardefs>
    <pardef n="numerals">
      <e>
        <p>
          <l>uno<s n="num"/></l>
          <r>um<s n="num"/></r>
        </p>
      </e>
      <e>
        <p>
          <l>uno<s n="det"/></l>
          <r>um<s n="det"/></r>
        </p>
      </e>
	\end{lstlisting}

	\column{.5\textwidth}
	{\bf With lt-expand:}
	\begin{center}
	\begin{verbatim}
	estar<vblex>:estar<vblex>
	decir<vblex>:dizer<vblex>
	tener<vblex>:ter<vblex>
	año<n>:ano<n>
	país<n>:país<n>
	obrar<vblex>:>:fazer<vblex>
	\end{verbatim}
	\vdots
	\begin{verbatim}
	primero<adj>:primeiro<adj>
	benéfico<adj>:benéfico<adj>
	primer<adj>:>:primeiro<adj>
	Irak<np>:Iraque<np>
	
	\end{verbatim}
	\end{center}
	\end{columns}
\end{frame}

\section{Usage}
\subsection{}
\begin{frame}[fragile]
	
	\begin{block}{Basic Usage}	
	\lstset{
		    		language=bash,
		    		basicstyle=\ttfamily\small,
		    		breaklines=true
				}
	\begin{lstlisting}
		$ apertium language-pair file
	\end{lstlisting}
	e.g.
	\lstset{
		    		language=bash,
		    		basicstyle=\ttfamily\small,
		    		breaklines=true
				}
	\begin{lstlisting}
		$ echo "Buenas tardes" apertium es-pt
	\end{lstlisting}
	Will produce the output:\\
	Boas Tardes
	\end{block}

\end{frame}

\subsection{}
\begin{frame}[fragile]
	After compilation of the dictionary we can analyse possible lexical forms.
	\begin{block}{Lexical Analysis}
	
	\lstset{
		    		language=bash,
		    		basicstyle=\ttfamily\small,
		    		breaklines=true
				}
	\begin{lstlisting}
		$ lt-proc morf-file file|word
	\end{lstlisting}
	\end{block}
	\begin{block}{Example}
	
	\lstset{
		    		language=bash,
		    		basicstyle=\ttfamily\small,
		    		breaklines=true
				}
	\begin{lstlisting}
		$ echo "prova" | lt-proc pt-es.automorf.bin
	\end{lstlisting}
	Will produce the following output:
	\lstset{
		    		language=xml,
		    		basicstyle=\ttfamily\small,
		    		breaklines=true
				}
	\begin{lstlisting}
		^prova/prova<n><f><sg>/provar<vblex><imp><p2><sg>/provar<vblex><pri><p3><sg>/provir<vblex><imp><p3><sg>/provir<vblex><prs><p1><sg>/provir<vblex><prs><p3><sg>$
	\end{lstlisting}
	\end{block}
\end{frame}

\begin{frame}
\titlepage
	\begin{center}
		\Huge\bfseries
		- ? -
	\end{center}
\end{frame}


\end{document}%	end presentation